%% abtex2-modelo-projeto-pesquisa.tex, v-1.9.2 laurocesar
%% Copyright 2012-2014 by abnTeX2 group at http://abntex2.googlecode.com/ 
%%
%% This work may be distributed and/or modified under the
%% conditions of the LaTeX Project Public License, either version 1.3
%% of this license or (at your option) any later version.
%% The latest version of this license is in
%%   http://www.latex-project.org/lppl.txt
%% and version 1.3 or later is part of all distributions of LaTeX
%% version 2005/12/01 or later.
%%
%% This work has the LPPL maintenance status `maintained'.
%% 
%% The Current Maintainer of this work is the abnTeX2 team, led
%% by Lauro César Araujo. Further information are available on 
%% http://abntex2.googlecode.com/
%%
%% This work consists of the files abntex2-modelo-projeto-pesquisa.tex
%% and abntex2-modelo-references.bib
%%

% ------------------------------------------------------------------------
% ------------------------------------------------------------------------
% abnTeX2: Modelo de Projeto de pesquisa em conformidade com 
% ABNT NBR 15287:2011 Informação e documentação - Projeto de pesquisa -
% Apresentação 
% ------------------------------------------------------------------------ 
% ------------------------------------------------------------------------

\documentclass[
	% -- opções da classe memoir --
	12pt,				% tamanho da fonte
	openright,			% capítulos começam em pág ímpar (insere página vazia caso preciso)
	oneside,			% para impressão em verso e anverso. Oposto a twoside
	a4paper,			% tamanho do papel.
	% -- opções da classe abntex2 --
	chapter=TITLE,		% títulos de capítulos convetidos em letras maiúsculas 
	%section=TITLE,		% títulos de seções convertidos em letras maiúsculas
	%subsection=TITLE,	% títulos de subseções convertidos em letras maiúsculas
	%subsubsection=TITLE,% títulos de subsubseções convertidos em letras maiúsculas
	% -- opções do pacote babel --
	english,			% idioma adicional para hifenização
	french,				% idioma adicional para hifenização
	spanish,			% idioma adicional para hifenização
	brazil,				% o último idioma é o principal do documento
	article,			% documento divido por sections
	]{uea-abntex2}

%\evensidemargin 0.5 cm

% ---
% PACOTES
% ---

% ---
% Pacotes fundamentais 
% ---
\usepackage{lmodern}			% Usa a fonte Latin Modern
\usepackage[T1]{fontenc}		% Selecao de codigos de fonte.
\usepackage[utf8]{inputenc}		% Codificacao do documento (conversão automática dos acentos)
\usepackage{indentfirst}		% Indenta o primeiro parágrafo de cada seção.
\usepackage{color}				% Controle das cores
\usepackage{graphicx}			% Inclusão de gráficos
\usepackage{microtype} 			% para melhorias de justificação
% ---

% ---
% Pacotes adicionais, usados apenas no âmbito do Modelo Canônico do abnteX2
% ---
\usepackage{lipsum}				% para geração de dummy text
% ---

% ---
% Pacotes de citações
% ---
\usepackage[brazilian,hyperpageref]{backref}	 % Paginas com as citações na bibl
\usepackage[alf]{abntex2cite}	% Citações padrão ABNT
\usepackage{tocloft}
\usepackage{parskip}

% --- 
% CONFIGURAÇÕES DE PACOTES
% ---

% ---
% Configurações do pacote backref
% Usado sem a opção hyperpageref de backref
\renewcommand{\backrefpagesname}{Citado na(s) página(s):~}
% Texto padrão antes do número das páginas
\renewcommand{\backref}{}
% Define os textos da citação
\renewcommand*{\backrefalt}[4]{
	\ifcase #1 %
		Nenhuma citação no texto.%
	\or
		Citado na página #2.%
	\else
		Citado #1 vezes nas páginas #2.%
	\fi}%
% ---

% ---
% Informações de dados para CAPA e FOLHA DE ROSTO
% ---
\titulo{SISTEMAS DE PROTEÇÃO CONTRA DESCARGAS ATMOSFÉRICAS}
\autor{FULANO DE TAL}
\local{Manaus}
\data{2014}
\instituicao{%
  UNIVERSIDADE DO ESTADO DO AMAZONAS
  \par
  ESCOLA SUPERIOR DE TECNOLOGIA}
\tipotrabalho{Projeto de Pesquisa}
\orientador[Orientador:]{Paulo Cavalcanti}
% O preambulo deve conter o tipo do trabalho, o objetivo, 
% o nome da instituição e a área de concentração 
\preambulo{Projeto de pesquisa proposto durante a disciplina Metodologia da
Pesquisa como pré-requisito para obtenção do título de Especialista em
Desenvolvimento de novos produtos pela Universidade do Estado do Amazonas,
Escola Superior de Tecnologia.}
% ---

% ---
% Configurações de aparência do PDF final

% alterando o aspecto da cor azul
\definecolor{blue}{RGB}{41,5,195}

% informações do PDF
\makeatletter
\hypersetup{
     	%pagebackref=true,
		pdftitle={\@title}, 
		pdfauthor={\@author},
    	pdfsubject={\imprimirpreambulo},
	    pdfcreator={LaTeX with abnTeX2},
		pdfkeywords={spda}{efeito corona}{descargas eletricas}, 
		colorlinks=false,       		% false: boxed links; true: colored links
   		linkcolor=black,          	% color of internal links
    	citecolor=blue,        		% color of links to bibliography
    	filecolor=magenta,      		% color of file links
		urlcolor=blue,
		bookmarksdepth=4
}
\makeatother
% --- 

% --- 
% Espaçamentos entre linhas e parágrafos 
% --- 

% O tamanho do parágrafo é dado por:
\setlength{\parindent}{1.3cm}

% Controle do espaçamento entre um parágrafo e outro:
\setlength{\parskip}{0.2cm}  % tente também \onelineskip

% ---
% compila o indice
% ---
\makeindex
% ---

% ----
% Início do documento
% ----
\begin{document}

% Retira espaço extra obsoleto entre as frases.
\frenchspacing 

% ----------------------------------------------------------
% ELEMENTOS PRÉ-TEXTUAIS
% ----------------------------------------------------------
% \pretextual

% ---
% Capa
% ---
\imprimircapa
% ---

% ---
% Folha de rosto
% ---
\imprimirfolhaderosto
% ---

% ---
% NOTA DA ABNT NBR 15287:2011, p. 4:
%  ``Se exigido pela entidade, apresentar os dados curriculares do autor em
%     folha ou página distinta após a folha de rosto.''
% ---

% ---
% inserir lista de ilustrações
% ---
%***********3 linhas comentadas pois não é obrigatória a Lista de Figuras
%\pdfbookmark[0]{\listfigurename}{lof}
%\listoffigures*
%\cleardoublepage
% ---

% ---
% inserir lista de tabelas
% ---
%***********3 linhas comentadas pois não é obrigatória a Lista de Tabelas
%\pdfbookmark[0]{\listtablename}{lot}
%\listoftables*
%\cleardoublepage
% ---

% ---
% inserir lista de abreviaturas e siglas
% ---
%\begin{siglas}
%  \item[ABNT] Associação Brasileira de Normas Técnicas
%  \item[abnTeX] ABsurdas Normas para TeX
%\end{siglas}
% ---

% ---
% inserir lista de símbolos
% ---
%\begin{simbolos}
%  \item[$ \Gamma $] Letra grega Gama
%  \item[$ \Lambda $] Lambda
%  \item[$ \zeta $] Letra grega minúscula zeta
%  \item[$ \in $] Pertence
%\end{simbolos}
% ---

% ---
% inserir o sumario
% ---
%\pdfbookmark[0]{\contentsname}{toc}
\tableofcontents*
%\cleardoublepage
% ---


% ----------------------------------------------------------
% ELEMENTOS TEXTUAIS
% ----------------------------------------------------------
\pagestyle{simple}

% ----------------------------------------------------------
% Introdução
% ----------------------------------------------------------
\textual
\newpage

\chapter*{\vspace*{3.4cm}INTRODUÇÃO}
\addcontentsline{toc}{section}{INTRODUÇÃO}

Conteudo da Introducao..
% ----------------------------------------------------------
% Capitulo de textual  
% ----------------------------------------------------------
\newpage
\chapter*{\vspace*{3.4cm}PROJETO DE PESQUISA}

\vspace{24pt}
\section{TEMA}

\section{FORMULAÇÃO DO PROBLEMA}

\section{HIPÓTESE}

\section{OBJETIVO}

\section{JUSTIFICATIVA}

\subsection{Justificativa Acadêmica}

\subsection{Justificativa Social}

\section{REFERENCIAL TEÓRICO}

\subsection{DISTRIBUIÇÃO DE ENERGIA ELÉTRICA}
\hspace*{0.8cm}O módulo 8 dos Procedimentos de Distribuição de Energia Elétrica
do Sistema Elétrico Nacional, PRODIST, obriga as concessionárias a medir e
controlar parâmetros referentes à qualidade da energia elétrica \cite{PRODIST}.
Desta forma, alguns dos diversos parâmetros de qualidade de energia que devem
ser mensurados são o nível de tensão, que deve ser mantido entre um máximo e um
mínimo e a continuidadeade do serviço, através dos parâmetros DEC (Duração
Equivalente por Consumidor) e FEC (Frequência Equivalente de Interrupção),
respectivamente. Entretanto, outros dois parâmetros são alvo de avaliação e
controle no sistema de distribuição: Forma de Onda e Simetria do Sinal, sendo
representados pelo indicador Taxa de Distorção Harmônica ou THD (\textit{Total
Harmonic Distortion}) dos sinais de tensão e corrente.

\subsection{A INFLUÊNCIA DE HARMÔNICOS}

\subsection{MEDIÇÃO DE HARMÔNICOS}

\subsection{MICROCONTROLADORES ARM}

\subsection{SISTEMA OPERACIONAL LINUX EMBARCADO}

\section{METODOLOGIA}

\hspace*{0.8cm}Serão feitas pesquisas bibliográficas na área de sistemas microprocessados de alto desempenho, com foco na arquitetura ARM com sistema operacional Linux embarcado, bem como também programação orientada a objeto, com foco na linguagem C++ e programação orientada a blocos, com base na ferramenta de desenvolvimento Labview, um ambiente de desenvolvimento da National Instruments que auxilia na confecção de interfaces gráficas. Também serão feitas pesquisas bibliográficas a respeito de métodos numéricos e algoritmos de cálculo da transformada rápida de Fourier. Serão, por fim, feitas pesquisas sobre circuitos de condicionamento, aplicados na leitura de sinais da rede elétrica, bem como de circuitos integrados medidores de energia, que serão utilizados para a amostragem dos sinais da rede a ser medida.

Pesquisas de campo serão aplicadas para coletar dados de redes de distribuição de energia elétrica, serão feitas, também, simulações computacionais e reais nas quais se buscará avaliar a confiabilidade dos algoritmos testados, bem como determinar o mais adequado às limitações inerentes à plataforma de trabalho disponível.

A construção do Sistema será dividida em duas etapas: A primeira etapa será a implementação dos algoritmos de cálculo de FFT, no Matlab, e comparação do desempenho e resposta dos mesmos com relação a blocos já prontos e disponíveis no referido software de simulação. Este método servirá como base de teste para a escolha do algoritmo de cálculo mais eficiente, em termos de processamento e exatidão.

A segunda etapa será a implementação de um protótipo utilizando do chip ADE7758, um circuito integrado medidor de energia e parâmetros elétricos trifásicos, para a aquisição das formas de onda de tensão e corrente, de tal modo que se possa implementar o algoritmo de cálculo de melhor desempenho em linguagem C++ em uma plataforma com sistema operacional Linux embarcado e comparar sua resposta com medidores comerciais utilizando uma interface de aquisição desenvolvida com o auxílio do software Labview.

Os resultados obtidos, bem como o algoritmo escolhido, ao final do projeto de pesquisa poderão ser incorporados a uma plataforma comercial de análise de rede elétrica utilizada no gerenciamento de transformadores de distribuição de energia elétrica, uma vez que os referidos recursos encontrem-se suficientemente testados em termos de eficiência e confiabilidade.


\newpage

\section{CRONOGRAMA}

As atividades de desenvolvimento do projeto seguirão o seguinte cronograma:
%\newpage
\begin{table}[h]\centering
\caption{Cronograma de atividades}
\begin{tabular}{|c|p{4cm}|c|c|c|c|c|c|c|c|r|}
\hline
\textbf{Item} & \textbf{Atividade} & \textbf{Abr} & \textbf{Mai} & \textbf{Jun} & \textbf{Jul} & \textbf{Ago} & \textbf{Set} & \textbf{Out} & \textbf{Nov} & \textbf{Dez} \\\hline \hline
1 & Escolha do Professor Orientador & X & & & & & & & & \\\hline
2 & Definição do Tema & & X & & & & & & & \\\hline
3 & Estudo sobre equipamentos de medição & & X & X & & & & & & \\\hline
4 & Coleta de Informações do projeto de pesquisa & & X & X & & & & & & \\\hline
5 & Tratamento das Informações coletadas & & & X & & & & & & \\\hline
6 & Elaboração do projeto de pesquisa & & & X & X & & & & & \\\hline
7 & Revisão do Texto do Projeto de Pesquisa & & & & X & & & & & \\\hline
8 & Elaboração da Apresentação & & & & X & & & & & \\\hline
9 & Apresentação do Projeto de Pesquisa & & & & X & & & & & \\\hline
10 & Escolha plataforma ARM & & & & & X & & & & \\\hline
11 & Análise de Dados de uma Rede Elétrica & & & & & X & X & & & \\\hline
12 & Comparação de Resultados de Cálculo & & & & & & X & & & \\\hline
13 & Elaboração textual da Pesquisa & & & & & & X & X & & \\\hline
14 & Revisão textual da pesquisa & & & & & & & X & & \\\hline
15 & Correções Textuais & & & & & & & & X & \\\hline
16 & Entrega da versão final da Pesquisa & & & & & & & & & X \\\hline
17 & Elaboração da Apresentação & & & & & & & & & X \\\hline
18 & Apresentação da Pesquisa & & & & & & & & & X \\\hline

\end{tabular}
\end{table}

%

% ----------------------------------------------------------
% Referências bibliográficas
% ----------------------------------------------------------
% Uncomment the following two lines if you want to have a bibliography. Please do not forget to add an entry to your bibliography and reference it by using the \cite{} command
\newpage
\vspace*{2.3cm}
\renewcommand{\bibname}{REFERÊNCIAS}
%\addcontentsline{toc}{section}{REFERÊNCIAS}
\bibliography{abntex2-modelo-references}

% ----------------------------------------------------------
% Glossário
% ----------------------------------------------------------
%
% Consulte o manual da classe abntex2 para orientações sobre o glossário.
%
%\glossary

% ----------------------------------------------------------
% Apêndices
% ----------------------------------------------------------


% ----------------------------------------------------------
% Anexos
% ----------------------------------------------------------


%---------------------------------------------------------------------
% INDICE REMISSIVO
%---------------------------------------------------------------------

\end{document}
